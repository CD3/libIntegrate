% arara: pdflatex
% arara: pdflatex

\documentclass[letterpaper,10pt]{article}
\usepackage{amsmath}
\usepackage{amsfonts}
\usepackage{amssymb}
\usepackage{graphicx}
\usepackage{siunitx}
\usepackage{physics}
\usepackage[left=1in,right=1in,top=1in,bottom=1in]{geometry}

\begin{document}

\section{Introduction}

Simpson's rule is a method for performing numerical integration. While there are several different approximations under the name ``Simpson's rule'', the most
common version, Simpson's 1/3 rule, is equivalent fitting a 2'nd degree polynomial to three points so that the integration can be carried out analytically.
It is usually written as:
\begin{equation}
  \label{eq:simpsons}
  \int_a^b f(x) \dd x \approx \frac{b-a}{6}\left[ f(a) + 4f\qty(\frac{a+b}{2}) + f(b)\right]
\end{equation}
This can be extended to larger domains with more function evaluations by breaking the domain into sub-domains and applying
\ref{eq:simpsons} to each.
However, equation \ref{eq:simpsons} assumes that the function $f(x)$ is evaluated at equally spaced points. If we have a discretized
function that is \emph{not} evaluated at equally spaced points, equation \ref{eq:simpsons} cannot be used.

\section{Generalizing to Non-Uniform Spacing}

To generalize equation \ref{eq:simpsons} to non-equally spaced points, we just need to perform the polynomial fit using non-equally spaced points an
perform the integration on the new fit. Using Lagrange polynomial interpolation, we have
\begin{equation}
  f(x) \approx f(x_1) \frac{ (x-x_2)(x-x_3) }{ (x_1 - x_2)(x_1 - x_3) }
             + f(x_2) \frac{ (x-x_1)(x-x_3) }{ (x_2 - x_1)(x_2 - x_3) }
             + f(x_3) \frac{ (x-x_1)(x-x_2) }{ (x_3 - x_1)(x_3 - x_2) }
\end{equation}
where $x_1$, $x_2$, and $x_3$ are the points at which the function is evaluated.

The Lagrange basis polynomials can be written as
\begin{equation}
  \label{eq:lagrange}
  l(x,A,B,C) = \frac{(x-A)(x-B)}{(C-A)(C-B)}
\end{equation}
and we can rewrite the interpolation as
\begin{equation}
  f(x) \approx
    f(x_1) l(x,x_2,x_3,x_1)
  + f(x_2) l(x,x_1,x_3,x_2)
  + f(x_3) l(x,x_1,x_2,x_3)
\end{equation}

\subsection{Method 1: Integrating the Lagrange polynomials directly}
Integrating the Lagrrange basis polynomials from $a$ to $b$ gives
\begin{equation}
  L(A,B,C,a,b) = \int_a^b l(x,A,B,C) \dd x = \frac{1}{(C-A)(C-B)} \left[ \frac{1}{3} \qty(b^3 - a^3) + \frac{A+B}{2}\qty(b^2 - a^2) + AB(b-a)  \right]
\end{equation}
and the integral of $f(x)$ is then approximated as
\begin{equation}
  \int_a^b f(x) \dd x \approx
    f(x_1) L(x_2,x_3,x_1,a,b)
  + f(x_2) L(x_1,x_3,x_2,a,b)
  + f(x_3) L(x_1,x_2,x_3,a,b)
\end{equation}

To reproduce equation \ref{eq:simpsons}, we let $x_1 = a$, $x_2 = \frac{a+b}{2}$, and $x_3 = b$. For example,
\begin{align}
  L(x_2,x_3,x_1,a,b) = \frac{1}{(a-\frac{a+b}{2})(a-b)} \qty[\frac{1}{3} \qty(b^3 - a^3) + \frac{\frac{a+b}{2} + b}{2}\qty(b^2 - a^2) + \frac{a+b}{2}b\qty(b-a) ]
\end{align}
apparently reduces to $\frac{b-a}{6}$...


\subsection{Method 2: Integrating the Lagrange polynomials by parts}

Integrating the Lagrange polynomials by parts gives
\begin{align}
  L(A,B,C,a,b) &= \int_a^b l(x,A,B,C) \dd x = \frac{1}{(C-A)(C-B)} \int_a^b (x - A)(x - B) \dd x  \\
  \dv{x} \qty[\frac{1}{2}(x-A)(x-B)^2] &= \frac{1}{2}(x - B)^2 + (x-A)(x-B) \\
  \int (x-A)(x-B) \dd x &= \frac{1}{2}(x-A)(x-B)^2 - \frac{1}{2} \int (x-B)^2 \dd x \\
  L(A,B,C,a,b) &= \left. \frac{1}{(C-A)(C-B)} \qty[ \frac{1}{2}(x-A)(x-B)^2 - \frac{1}{6} (x-B)^3] \right|_a^b
\end{align}

\subsection{Interpolating to uniform spacing}

Since Simpon's rule is derived by interpolating a function with a polynomial and integrating, we can use the polynomial to interpolate to the
medpoint of $[a,b]$ and just use the usual formula. Given the function at points $x_1$, $x_2$, and $x_3$, the interpolated function value at $\frac{x_1 + x_2}{2}$ is
\begin{equation}
  f\qty(\frac{x_1 + x_3}{2}) \approx 
  f(x_1) l( \frac{x_1+x_2}{2}, x_2,x_3,x_1)
  + f(x_2) l( \frac{x_1+x_2}{2}, x_1,x_3,x_2)
  + f(x_3) l( \frac{x_1+x_2}{2}, x_1,x_2,x_3)
\end{equation}
Once we interpolate to the midpoint, we can use equation \ref{eq:simpsons}.



\end{document}

